\chapter*{Introduction}
\addcontentsline{toc}{chapter}{Introduction}
Morphometry is a concept refers to qualitative analysis of form, it includes the size and the shape of an object. Morphometry analyses are commonly performed on organism. An objective of morphometry is to statistic hypotheses based on the effect of shape. Often, it was applied on the organism objects or the animal objects; but in scope of this field, we consider morphometry on biological images. Generally, to measure the object's morphometry, we can use one of three morphometry forms: traditional morphometry, landmark-based geometric morphometry, outline-based morphometry.  

In this work, we consider landmark-base geometric morphometry. Our goal is to implement the methods to extract the landmarks automatically from biological images and evaluate the results by statistic method. To archive the landmarks identification, the method of Palaniswamy\cite{palaniswamy2010automatic} has been chosen. The goal is to test if the landmarks provided by this method enough good that we can replace the manual landmarks. The application has been done on a set of beetle images which provided by biologists.

The whole report has four chapters. In the first chapter, this is the short introduction about the context working and the objectives of my internship. Chapter 2, introduces about the method which we use to extract the landmarks on biological images. In the third chapter, we mention how to apply the method for our problems. In the chapter 4, we will focus about the software. It introduces the architecture, the modules in the software. Besides, it also presents our examination, results and the summary about the future work.