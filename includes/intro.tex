\chapter*{Introduction}
\addcontentsline{toc}{chapter}{Introduction}
Morphometry is a concept refers to qualitative analysis of form, it includes the size and the shape of an object. Morphometry analyses are commonly performed on organism. An objective of morphometry is to statistic hypotheses based on the effect of shape. Normally, it was appied on the organism objects or the animal objects; but in scope of this field, we consider morphometry on biological images.

In image processing, morphometry can be used to detect the shape, size or changes in form on family organism.  Generally, to measure the object's morphometry, we can use one of three morphometry forms: traditional morphometry, landmark-based geometric morphometry, outline-based morphometry.  

In this internship, we consider landmark-base geometric morphometry. And our goal is to implement the methods to extract the landmarks automatically from biological images and evaluate the results by statistic method.

The whole report has four chapters. In the first chapter, this is the short introduction about the context working and the objectives of my internship. Chapter 2, introduces about the method which we use to extract the landmarks on biological images. In the third chapter, we mention how to apply the method for our problems. In the chapter 4, we will focus about the software. This chapter introduces the modules, the architecture and the packages used to build the software. Besides, we give our examination, results and the summary about the future work.