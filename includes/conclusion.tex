\chapter*{Conclusion}
\addcontentsline{toc}{chapter}{Conclusion}
\indent The landmarks are important characteristics used in shape analysis of many biological and medical applications. The landmarks help the user to indicate the shape of the object or classify the images follows the size of object.\\
After finishing the internship, we had done to implement the method \textbf{``Automatic identification of landmarks in digital images"} using OpenCV in C++. We also implemented the cross-correlation method and compare the results of two methods with the manual landmarks. Moreover, a new user interface (MAELAB) had constructed based on the IPM software. It had reorganization to suitable with the new requirements. The program also done with an examination on 2 set of images: \textit{left mandible} and \textit{right mandible}. The centroid from automated landmarks is the same with the centroid from the manual landmarks. We can use this property to classify the images. But, in some cases, the location of automated landmarks are not exactly with the manual landmarks. However, this method also a good method to estimate the landmark in biological image. A future work could be also to examine on other datasets.